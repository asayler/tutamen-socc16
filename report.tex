% Tutamen: A Second-Generation Secret Storage as a Service Platform
% Paper
%
% 2016
%
% Andy Sayler


\documentclass[letterpaper,twocolumn,10pt]{article}
\usepackage{usenix}

% System Packages
\usepackage{epsfig}
\usepackage{float}
\usepackage{caption}
\usepackage{subcaption}
\usepackage{tabu}
\usepackage{color}
\usepackage{hyperref}
\usepackage{url}

% Local Packages
% None

% Package Options
\hypersetup{
    colorlinks,
    citecolor=black,
    filecolor=black,
    linkcolor=black,
    urlcolor=black
}

% Macros
\newenvironment{packed_enum}{
\begin{enumerate}
  \setlength{\itemsep}{1pt}
  \setlength{\parskip}{0pt}
  \setlength{\parsep}{0pt}
}{\end{enumerate}}

\newenvironment{packed_item}{
\begin{itemize}
  \setlength{\itemsep}{1pt}
  \setlength{\parskip}{0pt}
  \setlength{\parsep}{0pt}
}{\end{itemize}}

\newenvironment{packed_desc}{
\begin{description}
  \setlength{\itemsep}{1pt}
  \setlength{\parskip}{0pt}
  \setlength{\parsep}{0pt}
}{\end{description}}


% Other Options
\clubpenalty = 10000
\widowpenalty = 10000

% Start
\begin{document}

%don't want date printed
\date{}

%make title bold and 14 pt font (Latex default is non-bold, 16 pt)
\title{\Large \bf Tutamen: A Second Generation Secret Storage Platform}

%for single author (just remove % characters)
\author{
{\rm Andy Sayler}\\
University of Colorado
\and
{\rm Taylor Andrews}\\
University of Colorado
\and
{\rm Dirk Grunwald}\\
University of Colorado
} % end author

\maketitle

% Use the following at camera-ready time to suppress page numbers.
% Comment it out when you first submit the paper for review.
%\thispagestyle{empty}

\subsection*{Abstract}

The storage and management of secrets (encryption keys, passwords,
etc) are significant opens problems in the modern age of cloud-based,
ephemeral computing infrastructure. How do we store and control access
to the secrets necessary to configure and operate a range of modern
technologies without sacrificing security and privacy requirements or
significantly curtailing the desirable capabilities of our
applications? As an answer to this question, we propose Tutamen - a
second generation secret-storage service. Tutamen has a number of
desirable secret storage properties including the ability to shard
secrets across multiple storage providers - ensuring both redundancy
and security, the ability to share secrets between users, and the
ability to require co textual and out-of-band authentication
parameters as a prerequisite to gaining secret access. These
properties allow Tutamen to be employed to achieve use cases not
easily realizable using existing systems such as providing autonomous
full-disk encryption to headless virtual machines or providing
client-side encryption for files stored atop cloud-based file lockers
or networked file systems while also supporting sharing and
multi-device access to such files. In this paper, we present the
nature of the secret storage problem, Tutamen's design and
architecture, the implementation of our Tutamen prototype, and several
of the applications in which we have implemented Tutamen-based secret
storage to achieve previously unattainable goals whiles till meeting a
variety of security and privacy requirements.

\section{Introduction}
\label{sec:intro}

How best to storage and manager secrets -- the bits of tightly
controlled data necessary to ensure or bootstrap and security of
larger data stores or computing systems -- has always been a
non-trivial problem. As we continue to move toward computing and
storage platforms controlled by third parties and embrace the ideals
of ephemeral infrastructure, the secret storage problems only become
more prevalent and important to answer. Tutamen\footnote{Latin -- A
  means of protection or defense.} is our attempt to solve the secret
storage problem in a manner that allows the user to adhere to a range
of security and privacy requirements.


\subsection{The Need for Secret Storage}

In response to the move toward ephemeral infrastructure, we have
developed and deployed numerous configuration management systems to
store the metadata required to deploy and operate a range of modern
technologies. System such as puppet~\cite{puppet} or chef~\cite{chef}
are designed to manage and configure large fleets of ephemeral
infrastructure. Such systems, however, do not tend to have suitable
mechanisms for storing and controlling access to configuration data
that must remain secret.

\subsection{Existing Solutions}

\cite{vault}
\cite{confidant}
\cite{openstack-barbican}
\cite{keywhiz}

\subsection{Desirable Properties}

\section{The Tutamen Platform}
\label{sec:tutamen}

\subsection{Architecture}

\subsection{Security and Trust}

\subsection{Implementation}

\section{Applications}
\label{sec:apps}

\subsection{Server Disk Encryption}

\subsection{Encrypted File Locker}

\section{Evaluation}
\label{sec:eval}

\subsection{Usage}

\section{Conclusion}
\label{sec:conclusion}

\subsection{Future Work}

{
  \footnotesize
  \bibliographystyle{acm}
  \bibliography{refs}
}

\end{document}

%%  LocalWords:  Tutamen Tutamen's
