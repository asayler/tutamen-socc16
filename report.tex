% Tutamen: A Second-Generation Secret Storage as a Service Platform
% Paper
%
% 2016
%
% Andy Sayler


\documentclass[letterpaper,twocolumn,10pt]{article}
\usepackage{usenix}

% System Packages
\usepackage{epsfig}
\usepackage{float}
\usepackage{caption}
\usepackage{subcaption}
\usepackage{tabu}
\usepackage{color}
\usepackage{hyperref}
\usepackage{url}

% Local Packages
% None

% Package Options
\hypersetup{
    colorlinks,
    citecolor=black,
    filecolor=black,
    linkcolor=black,
    urlcolor=black
}

% Macros
\newenvironment{packed_enum}{
\begin{enumerate}
  \setlength{\itemsep}{1pt}
  \setlength{\parskip}{0pt}
  \setlength{\parsep}{0pt}
}{\end{enumerate}}

\newenvironment{packed_item}{
\begin{itemize}
  \setlength{\itemsep}{1pt}
  \setlength{\parskip}{0pt}
  \setlength{\parsep}{0pt}
}{\end{itemize}}

\newenvironment{packed_desc}{
\begin{description}
  \setlength{\itemsep}{1pt}
  \setlength{\parskip}{0pt}
  \setlength{\parsep}{0pt}
}{\end{description}}


% Other Options
\clubpenalty = 10000
\widowpenalty = 10000

% Start
\begin{document}

%don't want date printed
\date{}

%make title bold and 14 pt font (Latex default is non-bold, 16 pt)
\title{\Large \bf Tutamen: A Second Generation Secret Storage Platform}

%for single author (just remove % characters)
\author{
{\rm Andy Sayler}\\
University of Colorado
\and
{\rm Taylor Andrews}\\
University of Colorado
\and
{\rm Dirk Grunwald}\\
University of Colorado
} % end author

\maketitle

% Use the following at camera-ready time to suppress page numbers.
% Comment it out when you first submit the paper for review.
%\thispagestyle{empty}

\subsection*{Abstract}

The storage and management of secrets (encryption keys, passwords,
etc) are significant opens problems in the modern age of cloud-based,
ephemeral computing infrastructure. How do we store and control access
to the secrets necessary to configure and operate a range of modern
technologies without sacrificing security and privacy requirements or
significantly curtailing the desirable capabilities of our
applications? As an answer to this question, we propose Tutamen - a
second generation secret-storage service. Tutamen has a number of
desirable secret storage properties including the ability to shard
secrets across multiple storage providers - ensuring both redundancy
and security, the ability to share secrets between users, and the
ability to require co textual and out-of-band authentication
parameters as a prerequisite to gaining secret access. These
properties allow Tutamen to be employed to achieve use cases not
easily realizable using existing systems such as providing autonomous
full-disk encryption to headless virtual machines or providing
client-side encryption for files stored atop cloud-based file lockers
or networked file systems while also supporting sharing and
multi-device access to such files. In this paper, we present the
nature of the secret storage problem, Tutamen's design and
architecture, the implementation of our Tutamen prototype, and several
of the applications in which we have implemented Tutamen-based secret
storage to achieve previously unattainable goals whiles till meeting a
variety of security and privacy requirements.

\section{Introduction}
\label{sec:introduction}

This is an example section.

\subsection{Subsection A}
\label{sec:section1A}

This is an example subsection

{
  \footnotesize
  \bibliographystyle{acm}
  \bibliography{refs}
}

\end{document}

%%  LocalWords:  Tutamen Tutamen's
