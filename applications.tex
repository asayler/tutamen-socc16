\section{Example Applications}
\label{sec:apps}

Tutamen is designed to support a wide range of applications. We have
integrated our reference Tutamen design with a set of common
applications for the purpose of demonstrating the value derived from
using a secure storage system such as Tutamen. These applications all
leverage Tutamen's flexibility to achieve functionality that would
have been difficult or impossible to achieve without using a system
like Tutamen.

\subsection{Block Device Encryption}

Block device encryption systems are a popular means of protecting the
data stored on computing systems in the event that the system is lost,
stolen, or otherwise physically compromised.  Block-level encryption
systems such as dm-crypt~\cite{dm-crypt} (generally coupled with the
Linux Unified Key Setup (LUKS)~\cite{luks} container) or the
QEMU~\cite{qemu} qcow2 encryption system provide methods for securing
the data stored on laptops, desktops, and VMs. Such systems
traditionally bootstrap security by requiring the user to enter a
password at boot-time to unlock a locally stored encryption key which
is then used to decrypt the block device in question. Unfortunately,
the ``human-at-the-keyboard'' security root make such systems
difficult or impossible to use atop headless servers or in situations
where no human can be expected to be present at boot-time. We have
leveraged Tutamen to overcome this barrier.

To add Tutamen-support to LUKS/dm-crypt we have integrated Tutamen
with the LUKS/dm-crypt bootstrapping
process~\cite{src-tutamen-askpassword}. Our integration replaces the
traditional ``human-at-the-keyboard'' boot-time password prompt with a
request to a Tutamen storage server for necessary decryption secret
(after first retrieving the necessary tokens from the corresponding
Tutamen AC server). We have also integrated Tutamen support with QEMU
to provide qcow2 encryption keys when a VM
boots~\cite{src-qemu-tutamen}. Similar to the dm-crypt setup, QEMU
normally requires the user to provide the encryption key via the QEMU
console when a VM launches. Our system replaces this process with
Tutamen-based secret retrieval. Using these setups, we're able to boot
servers and VM images with encrypted disks without requiring a human
to be physically present at the machine. In cases where we still
desire human approval of the boot process, we can leverage our SMS
authenticator module to get an on-demand confirmation from a
designated human as a prerequisite to Tutamen releasing the correct
key. This allows us to gain the same level of human-in-the-loop
security provided by a typed passphrase, but without actually
requiring a human to go to the datacenter to type one in. In
situations where we don't desire a human-in-the-loop at all, we
envision automating the approval process via the use of time-of-day
and IP-source authenticators.

\subsection{Encrypted Cloud Storage}

Cloud-based file storage systems such as Dropbox~\cite{dropbox} are
extremely popular today. Unfortunately, these systems require users to
trust the cloud provider with full access to their (generally
unencrypted) data. Users wishing to overcome this deficiency can
optionally encrypt all of their data on the client before syncing it
to the file locker provider, but doing so does not generally interact
well with such services' sharing and multi-device use cases, requiring
users to employ manual, out-of-band key exchange mechanisms to share
or sync their encrypted files. We don't believe file locker users
should have to choose between easily syncing or sharing their files
and using encryption to protect their data. Tutamen provides a
solution to this problem by offering a secure key-sharing
mechanism. Instead of manually distributing or sharing encryption
keys, the user can store their key as a Tutamen secret and leverage
Tutamen's access control features to share the secret with the
accounts of their friends. This entire process can even be automated
such that when a user shares a file via Dropbox, the corresponding
encryption key is automatically shared via Tutamen.

Toward this end, we have created FuseBox~\cite{fusebox}: an alternate
Dropbox client that performs client-side encryption of all Dropbox
files, storing the corresponding encryption keys on our reference
Tutamen server. FuseBox achieves goals similar to those achieved
by~\cite{goh2003}, but without requiring out-of-band key
management. Similar to other file-system-level encryption
systems~\cite{blaze1993, Cattaneo2001, halcrow}, FuseBox provides
transparent file encryption to end users. In order to avoid the
storage space and security challenges presented by locally caching all
Dropbox data (i.e., the operation mode for the official Dropbox
client), FuseBox uses AES~\cite{daemen1999, nist2001} as a stream
cipher to transparently stream and encrypted data to/from Dropbox's
servers on demand. Since FuseBox leverages Tutamen to store each
per-file encryption key, FuseBox natively supports Dropbox's
multi-device sync use case by allowing a user to access and decrypt
their Dropbox files from any device on which they have setup the
Tutamen client. FuseBox also makes it simple to share an encrypted
file via Dropbox, share its encryption key via Tutamen, and achieve
the same level of functionality traditional Dropbox users have without
having to expose one's data to Dropbox.\footnote{While the key sharing
  process in FuseBox is not yet directly synced with Dropbox's file
  sharing system, the Tutamen CLI can be used to quickly share the
  encryption keys between users.} In this manner, we have used FuseBox
to store and share encrypted files with nearly the same ease with
which one might use the traditional unencrypted Dropbox client. By
leveraging Tutamen, FuseBox also gains the ability to remotely revoke
file access, e.g., in the case a device is lost or stolen, similar to
systems such as~\cite{geambasu2011}. FuseBox, via Tutamen's
distributed operation mode, also avoids the sharing pitfalls
associated many existing ``secure cloud storage''
providers~\cite{wilson2014} by avoiding reliance on a single trusted
party to facilitate sharing operations.

%%  LocalWords:  Tutamen keyslots keyslot systemd cryptsetup initrd
%%  LocalWords:  libcryptsetup initramfs Archlinux RedHat ini pid FDE
%%  LocalWords:  dgram tutamen OpenVPN TUTATEM luks authenticators dm
%%  LocalWords:  SMS authenticator golang Tutamen's FuseBox FuseBox's
%%  LocalWords:  QEMU qcow
