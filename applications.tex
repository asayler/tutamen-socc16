\section{Applications}
\label{sec:apps}

Tutamen is designed to support a wide range of applications. We have
integrate our reference Tutamen design with a set of common
applications for the purpose of demonstrating the value derived from
using a secure storage system such as Tutamen. These applications all
achieve functionality that would have been difficult or impossible to
achieve without using a system like Tutamen.

\subsection{Full Disk Encryption}

Full disk encryption (FDE) system are a popular means of protecting
the data stored on computing systems in the event that the system is
lost, stolen, or otherwise physically compromised. In Linux,
block-level encryption system such as dm-crypt~\cite{dm-crypt}
(generally coupled with the Linux Unified Key Setup (LUKS)~\cite{luks}
container) provide a popular method for securing the data stored
laptops, desktops, and portable storage devices. Such systems
generally require the user to enter a password at boot-time to unlock
a locally stored encryption key that is then used to decrypt the block
device in question. Unfortunately, the ``human-at-keyboard security''
root make such systems difficultly or impossible to use atop headless
servers or automated system where there no human can be expected to be
present at boot time.

In the standard dm-crypt FDE deployment, the LUKS header specifies
which encryption algorithms to use and contains the master symmetric
encryption key. The master key is itself encrypted and stored in one
or more of with available ``keyslots'' in the LUKS header where each
``keyslot'' corresponds to a specific pass-phrase capable of
decrypting the master-key. A pass-phrase to decrypt one of the copies
of is provided at boot-time via the LUKS cryptsetup utility (which is
installed in the local initramfs image). This utility is capable of
obtaining the pass-phrase form a variety of sources, but most commonly
simply quires the user to enter the pass-phrase
directly\footnote{other modes of operation involve reading the
  pass-phrase off an external storage device temporally connected to
  the machine or extracting it from a local TPM} -- introducing an
undesirable ``human-at-keyboard'' dependency in the case of most
servers.

The cryptsetup utility relies on the systemd Password Agent
Specification~\cite{systemd-passwordagents} to request a pass-phrase
from the user. We've created a Tutamen-aware implementation of this
specification capable of responding to cryptsetup pass-phrase
requests. Using this utility, cryptsetup either prompt the user for a
pass-phrase directly or request the pass-phrase from a Tutamen storage
server (after first retrieving the necessary tokens from the
corresponding Tutamen AC server). In addition to modifying the
ask-password utility, we made several modifications to the initrd
creation process to add Tutamen networking support, the necessary
Tutamen client TLS key pair, and a config file specifying which
Tutamen servers to use, and the UUIDs of the relevant Tutamen
collection and secrets).

Using this setup, we're able to boot servers with encrypted root (or
other disks) without requiring a human to be physically present at the
machine. In cases where we still desire human confirmation of such
boot setups, we can leverage our SMS authenticator module to get an
on-demand confirmation form a human anywhere they have cellular
service as a prerequisite to Tutamen releasing the correct key. This
allows us to gain the same level of human-in-the-loop security
provided by a typed pass-phrase but without actually requiring a human
to go to the data center to type one in. In situations where we don't
desire a human-in-the-loop at all, we envision the automate the
approval process via the use of Tutamen time-of-day and IP-source
authenticators or more advanced authenticators that approve or deny
request based on their environmental factors and the system's history
of previous requests.

Our Tutamen-capable ask-password port and supporting initramfs code is
available at~\cite{src-tutamen-askpassword}. This code leverages a
Tutamen golang library we created to allows programs written in Go to
request secrets from Tutamen servers~\cite{src-tutamen-go}. Since this
utility speaks the standard systemd Password Agent protocol, it can
also be used to provide Tutamen-backed pass-phrase storage to any
applications leveraging this protocol, including OpenVPN and other
standard system utility. We've not yet explored these sue cases as
thoroughly, but we envision a Tutamen-passed systemd password agent
being useful in a wide range of situations beyond just full disk
encryption.

%% Each header contains eight available keyslots in the which
%% can be set by an admin. The 0th keyslot is setup at creation time and
%% manipulating the others and dumping the master key require access to
%% just 1 of the slots.

%% systemd-cryptsetup is linked against libcryptsetup, the same library that powers
%% the normal cryptsetup utility for LUKS. It is built into the initrd/initramfs and
%% run during boot in order to initialize root. How it is copied in and how it is
%% executed is distribution dependent, I can elaborate on Archlinux and RedHat...
%% It is configured via rd.luks.*= kernel command line parameters, namely
%% rd.luks.name=<UUID>=<name> tells it to map a LUKS block device with UUID to
%% /dev/mapper/<name>.

%% systemd-cryptsetup (as you might expect from a binary in the systemd repository)
%% implements the systemd Password Agents specification
%% http://www.freedesktop.org/wiki/Software/systemd/PasswordAgents/. An ini file
%% is written to /run/systemd/ask-password/ask.<randomstring> when asking for a
%% password. This file contains info such as the pid of the calling process, an
%% optional id, and a mandatory socket. Any program running on the system can
%% monitor this directory and respond to password requests by writing simple
%% strings into the Unix domain dgram socket.

%% One such program is systemd-ask-password-console which writes a password prompt
%% to the console and waits for human input. This will be shown on the screen
%% during the normal boot process

%% We added tutamen-ask-password in go which can respond to \textbf{any password
%% agent request} including the one by systemd-cryptsetup. Another notable one is
%% built into OpenVPN. We also added a bit of Archlinux-tooling to get
%% tutamen-ask-password and into the initrd. It copies the entire tutamen config
%% directory pointed to by TUTATEM\_CONFIG into /root/.config on the initrd. The
%% collection and secret UUIDs are read from the tutamen ini files as well.
%% tutamen-ask-password does not interrupt the normal flow of the out of the box
%% console agent. A user can still enter a password manually before the network/tutamen
%% has a chance to run, while it's running, or if it fails.

%% We could add a tool to generate and add a key to luks and store in tutamen in
%% one go

%% Unrelated but since some authenticators implicitly provide some logging
%% features, you can talk about a log\_noop authenticator or such if you need to
%% fill some space.


\subsection{VM Image Encryption}

\subsection{Encrypted File Locker}
An encrypted file locker protects data such that only those with a
cryptographic key can view anything. Traditionally, the interaction 
between user and file locker happens on a single machine with a single 
key. Cloud computing services seek to break out of this mold. For example,
Dropbox is a file locker that supports many sharing features. Files
can be easily accessed from a different machine, or easily shared with
another user. If anyone were to upload an encrypted file they
would be forced to copy and manually share their key with any other
user or machine that needed access to the file. Users shouldn't have
to sacrifice the security of a encrypted file locker to use the
sharing features packaged with Dropbox. 

The solution to this problem is secure key sharing. Tutamen provides
this functionality in a flexible and secure way. Instead of tediously
copying keys, the user can store their key in a Tutamen
secret. Then all they have to do is share that secret with any desired
party. Tutamen also provides collections. This allows users to group
together a batch of files, and then share access to all of them at
once. 

It may be a bit of a challenge for the average user to successfully
combine using Dropbox, encryption, and Tutamen. To that end we
created FuseBox. The source code is available here~\cite{fusebox}.
FuseBox is an encrypted file locker that leverages Tutamen to share
files between trusted parties. The application sits transparently 
between the end user and Dropbox. In order to decrease its memory
footprint on the user,
FuseBox adopts an on the fly approach that differentiates it from the
default Dropbox client. Files are retrieve from Dropbox only as
needed. This sacrifices offline capability for taking up less space. 

%%  LocalWords:  Tutamen keyslots keyslot systemd cryptsetup initrd
%%  LocalWords:  libcryptsetup initramfs Archlinux RedHat ini pid FDE
%%  LocalWords:  dgram tutamen OpenVPN TUTATEM luks authenticators dm
%%  LocalWords:  SMS authenticator golang
