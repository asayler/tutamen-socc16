\section{Applications}
\label{sec:apps}

Tutamen is designed to support a wide range of applications. We have
integrated our reference Tutamen design with a set of common
applications for the purpose of demonstrating the value derived from
using a secure storage system such as Tutamen. These applications all
leverage Tutamen's flexibility to achieve functionality that would
have been difficult or impossible to achieve without using a system
like Tutamen.

\subsection{Full Disk Encryption}

Full disk encryption (FDE) system are a popular means of protecting
the data stored on computing systems in the event that the system is
lost, stolen, or otherwise physically compromised. In Linux,
block-level encryption systems such as dm-crypt~\cite{dm-crypt}
(generally coupled with the Linux Unified Key Setup (LUKS)~\cite{luks}
container) provide a popular method for securing the data stored on
laptops, desktops, and portable storage devices. Such systems
traditionally require the user to enter a password at boot-time to
unlock a locally stored encryption key that is then used to decrypt
the block device in question. Unfortunately, the ``human-at-keyboard
security'' root make such systems difficult or impossible to use atop
headless servers or other automated system where no human can be
expected to be present at boot time.

In the standard dm-crypt FDE deployment, the LUKS header specifies
which encryption algorithms to use and contains the master symmetric
encryption key. The master key is itself encrypted and stored in one
or more of the available ``keyslots'' in the LUKS header, where each
``keyslot'' corresponds to a specific pass-phrase capable of
decrypting the master-key. Traditionally, the user of such a system is
prompted to enter one of these pass phrases at boot time via the LUKS
cryptsetup utility (which is installed in the local initramfs image)
-- introducing an undesirable ``human-at-keyboard'' dependency in the
server or datacenter use case.

The cryptsetup utility relies on the systemd Password Agent
Specification~\cite{systemd-passwordagents} to request a pass-phrase
from the user. We've created a Tutamen-aware implementation of this
specification capable of responding to cryptsetup pass-phrase
requests. Using this utility, cryptsetup can either prompt the user
for a pass-phrase directly, or request the pass-phrase from a Tutamen
storage server (after first retrieving the necessary tokens from the
corresponding Tutamen AC server). In addition to modifying the
ask-password utility, we made several modifications to the initrd
creation process to add Tutamen networking support, the necessary
Tutamen client TLS key pair, a config file specifying which
Tutamen servers to use, and the UUIDs of the relevant Tutamen
collection and secrets.

Using this setup, we're able to boot servers with encrypted root (or
other) disks without requiring a human to be physically present at the
machine. In cases where we still desire human approval of the boot
process, we can leverage our SMS authenticator module to get an
on-demand confirmation from a designated human anywhere they have
cellular service as a prerequisite to Tutamen releasing the correct
key. This allows us to gain the same level of human-in-the-loop
security provided by a typed pass-phrase but without actually
requiring a human to go to the datacenter to type one in. In
situations where we don't desire a human-in-the-loop at all, we
envision an automated the approval process via the use of Tutamen
time-of-day and IP-source authenticators, or more advanced
authenticators that leverage machine learning techniques to approve or
deny requests based on the system's history of previous requests.

Our Tutamen-capable ask-password port and supporting initramfs
configuration code is available
at~\cite{src-tutamen-askpassword}. Since this utility speaks the
standard systemd Password Agent protocol, it can also be used to
provide Tutamen-backed pass-phrase storage to any applications
leveraging this protocol, including OpenVPN and other standard system
utility. We've not yet thoroughly explored these use cases, but we
envision a Tutamen-passed systemd password agent being useful in a
wide range of situations beyond just full disk encryption.

%% Each header contains eight available keyslots in the which
%% can be set by an admin. The 0th keyslot is setup at creation time and
%% manipulating the others and dumping the master key require access to
%% just 1 of the slots.

%% systemd-cryptsetup is linked against libcryptsetup, the same library that powers
%% the normal cryptsetup utility for LUKS. It is built into the initrd/initramfs and
%% run during boot in order to initialize root. How it is copied in and how it is
%% executed is distribution dependent, I can elaborate on Archlinux and RedHat...
%% It is configured via rd.luks.*= kernel command line parameters, namely
%% rd.luks.name=<UUID>=<name> tells it to map a LUKS block device with UUID to
%% /dev/mapper/<name>.

%% systemd-cryptsetup (as you might expect from a binary in the systemd repository)
%% implements the systemd Password Agents specification
%% http://www.freedesktop.org/wiki/Software/systemd/PasswordAgents/. An ini file
%% is written to /run/systemd/ask-password/ask.<randomstring> when asking for a
%% password. This file contains info such as the pid of the calling process, an
%% optional id, and a mandatory socket. Any program running on the system can
%% monitor this directory and respond to password requests by writing simple
%% strings into the Unix domain dgram socket.

%% One such program is systemd-ask-password-console which writes a password prompt
%% to the console and waits for human input. This will be shown on the screen
%% during the normal boot process

%% We added tutamen-ask-password in go which can respond to \textbf{any password
%% agent request} including the one by systemd-cryptsetup. Another notable one is
%% built into OpenVPN. We also added a bit of Archlinux-tooling to get
%% tutamen-ask-password and into the initrd. It copies the entire tutamen config
%% directory pointed to by TUTATEM\_CONFIG into /root/.config on the initrd. The
%% collection and secret UUIDs are read from the tutamen ini files as well.
%% tutamen-ask-password does not interrupt the normal flow of the out of the box
%% console agent. A user can still enter a password manually before the network/tutamen
%% has a chance to run, while it's running, or if it fails.

%% We could add a tool to generate and add a key to luks and store in tutamen in
%% one go

%% Unrelated but since some authenticators implicitly provide some logging
%% features, you can talk about a log\_noop authenticator or such if you need to
%% fill some space.


\subsection{VM Image Encryption}

{\em Need to add this section and/or combine it with the FDE section above}

\subsection{Encrypted File Locker (Dropbox)}

Cloud-based file lockers such as Dropbox~\cite{dropbox} are extremely
popular today. Unfortunately, these systems require users to trust the
cloud provider with full access to their (generally unencrypted)
data. Users wishing to overcome this deficiency can optionally encrypt
all of their data on the client before syncing it to the file locker
provider, but doing so does not generally interact well with such
services' sharing and multi-device use cases. Client-side encryption
traditionally requires users to employ manual, out-of-band key
exchange mechanisms to share or sync their encrypted files. We don't
believe file locker users should have to choose between easily syncing
or sharing their files and using encryption to protect their data.

Tutamen provides a solution to this problem by offering a secure
key-sharing mechanism. Instead of manually distributing or sharing
encryption keys, the user can store their key as a Tutamen secret and
leverage Tutamen's access control features to share the secret with
the accounts of their friends. This entire process could even be
automated such that when a user shares a file via Dropbox the
corresponding encryption key is automatically shared via Tutamen.

Toward this end, we have created FuseBox: an alternate Dropbox client
that performs client-side encryption of all Dropbox files, storing the
corresponding encryption keys on our reference Tutamen server. The
source code for our FuseBox implementation is available
at~\cite{fusebox}.  FuseBox transparently encrypts and decrypts files
when they are opened and closed -- streaming the encrypted file data
to or from Dropbox as necessary. Since FuseBox leverages Tutamen to
store each per-file encryption key, it becomes possible to share an
encrypted file via Dropbox, share its encryption key via Tutamen, and
achieve the same level of functionally traditional Dropbox users have
without having to expose one's data to Dropbox. While the key sharing
process in FuseBox is not yet directly synced with Dropbox's file
sharing system, the Tutamen CLI can be used to quickly share the
encryption keys between users. In this manner, we've used FuseBox to
store and share encrypted files with nearly the same ease with which
one might use the traditional unencrypted Dropbox client.

%%  LocalWords:  Tutamen keyslots keyslot systemd cryptsetup initrd
%%  LocalWords:  libcryptsetup initramfs Archlinux RedHat ini pid FDE
%%  LocalWords:  dgram tutamen OpenVPN TUTATEM luks authenticators dm
%%  LocalWords:  SMS authenticator golang Tutamen's FuseBox
