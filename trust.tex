\section{Security and Trust}
\label{sec:trust}

One of Tutamen's primary design goals is its ability to support a wide
range of security and trust requirements. In this section, we present
a basic overview of the Tutamen security and trust models.

\subsection{Security of Individual Servers}

The security of each individual access control server rests on several
requirements. Failure to uphold these requirements will result in the
failure of any security guarantees provided by the AC server.

\begin{packed_desc}
\item[Certificate Authority Role:] Each access control server acts as
  a CA delegated with issuing and verifying client certificates. Thus,
  each AC server must store its CA keys in a secure manner and
  faithfully verify the certificate presented by each client
  connection.
\item[Token Issuance and Verification:] Each access control server is
  responsible for verifying the access control requirements bound to
  specific object/permission combinations, issuing signed tokens
  attesting to such verification, and verifying the signatures of the
  tokens it receives from clients wishing to operate on access control
  objects. Thus, each AC server must store private token signing key
  in a secure manner and faithfully verify both the access control
  requirements governing specific token requests as well as the
  signatures on all incoming tokens.
\end{packed_desc}

The storage servers must uphold the following security
requirements. Failure to do so results in a failure of the security of
the storage server.

\begin{packed_desc}
\item[Token Verification:] Each storage server must securely (via
  HTTPS) obtain the public token signing key from each AC server
  delegated with providing access control for a given storage
  object. The storage server must then use these keys to faithfully
  verify the signatures on all tokens it receives. Assuming the token
  signature is valid, the storage server must faithfully enforce the
  claims asserted in a given token; e.g., by only allowing actions
  granted by the permission contained in the token on the object the
  token identifies prior to the expiration time specified by the
  token.
\item[Secure Storage:] Each storage server must take steps to store
  user-provided secrets in a secure manner, releasing them only to
  requests accompanied by the requisite number of valid tokens
  granting such release.
\end{packed_desc}

Since the tokens the storage server must verify are provided by the AC
servers, the security of the storage server with respect to a given
collection is dependent on the security of any designated AC servers
associated with said collection. If these AC servers are insecure, the
objects that delegate access to them will also be
insecure.\footnote{Since the security of Tutamen is derived from the
  security of the access control server, it is reasonable for a single
  host to operate both access control and storage servers. Such a
  deployment requires no more trust than a host who operates only an
  access control server. The Tutamen access control and storage roles
  are mainly split into separate servers to allow for independent
  scaling of each role and to promote separation of duties in the code
  base, and collocating both servers types is not generally
  detrimental to the security of the system.}

\subsection{Security of Multiple Servers}

Unlike existing secret management systems~\cite{vault, confidant,
  keywhiz}, the Tutamen architecture is designed to support users
outside of a single administrative domain and is capable of remaining
secure even when individual storage or access control servers fail to
meet their security requirements. Such failures may result from
physical server compromise, software bugs, malicious intent or
incompetence on the part of the server operator, or compelled
failures.\footnote{For example, being forced to turn over stored
  secrets in response to legal or governmental pressure.}

To work around security failures of individual server, Tutamen
applications can leverage Tutamen's distributed operation modes. In
these modes, the security of the system as a whole is diffused, no
longer relying on the security of any specific access control or
storage server in order to keep an application's secrets secure. As
described in Section~\ref{sec:tutamen:arch:distributed}, each
application can distribute both secret-storage and access control
delegations using $n$ choose $k$ schemes. In such setups, the value of
$k$ represents the degree to which a Tutamen applications can
withstand security failures of the associated AC and storage servers,
while the difference between $n$ and $k$ dictates the degree to which
an Application can withstand availability failures. For example, an
application which chooses to shard its secrets across six storage
servers where any three shards are sufficient to recreate the secret
($n=6$, $k=3$) will remain secure as long as no more than two ($k-1$)
of the storage servers fail to meet their security
obligations. Similarly, if each secret shard delegates six possible AC
servers, tokens from three of which are required to grant secret
access, the applications can withstand the failure of two AC servers
to uphold their security guarantees. In both cases, the system can
also withstand the availability failure of up to three servers ($n-k$)
while continuing to operate.

\subsection{Trust Model}

Trust in Tutamen follows from the security models of both individual
Tutamen servers and of the distributed Tutamen deployment
architectures. If a Tutamen application is leveraging only a single
storage and AC server, the application is placing a high degree of
trust in both servers (and by proxy, the operators of both
servers). This level of trust may be appropriate for some use cases
(e.g., when a user is operating their own Tutamen's servers), but is
inappropriate in many other cases (e.g., when using third party hosted
Tutamen servers). Fortunately, Tutamen allows applications to avoid
placing a high degree of trust in any single server by leveraging
multiple servers and picking $k$ and $n$ in a manner commensurate with
the degree to which each server is trusted.

When selecting Tutamen hosts for distributed Tutamen operation, it is
desirable to select hosts with geospatial, geopolitical, and
administrative diversity. Doing so reduces the likelihood that
multiple servers will be subject to the same failure (e.g. a regional
power outage) and also reduces the likelihood of collusion (e.g. by
avoiding the use of multiple Tutamen hosting providers controlled by a
single administrative entity) as well as the likelihood of compelled
access (e.g. by locating secret shards across national boundaries
where no single government can compel access to all shards). Should a
system such as Tutamen ever become standardized, we also envision the
formation of a competing market of Tutamen providers from which users
may select individual hosts for their secrets. Such a market has the
potential to align economic benefits with good-faith efforts by server
providers to uphold the Tutamen security
guarantees~\cite{sayler-phd}. Although using Tutamen will also entail
risks with respect to trusted third parties, we believe these risks
are much lower than existing centralized secret storage platforms and
that increasing the number of providers over which secrets or access
control duties are split serves to arbitrarily reduce the degree of
such risks.

%%  LocalWords:  Tutamen's Tutamen CAs DNS TPMs authenticators HTTPS
%%  LocalWords:  ACS
