\section{Security and Trust}
\label{sec:trust}

One of Tutamen's primary design goals is its ability to support a wide
range of security and trust requirements. It achieves this goal
through its support for both centralized and distributed operation as
well as though its support for a range of authentication mechanisms.

\subsection{Security of Individual Servers}

The security of each individual access control server rests on several
requirements. Failure to uphold these requirements will result in the
failure of any security guarantees provided by the AC server.

\begin{packed_desc}
\item[Certificate Authority Role:] Each access control server acts as
  a CA delegated with issuing and verifying client certificates. Thus,
  each AC server must store its CA keys in a secure manner and
  faithfully verify the certificate presented by each client
  connection.
\item[Token Issuance and Verification:] Each access control server is
  responsible for verifying the access control requirements bound to
  specific object/permission combinations, issuing signed tokens
  attesting to such verification, and verifying the signatures of the
  tokens it receives from clients wishing to operate on access control
  objects. Thus, each AC server must store private token signing key
  in a secure manner and faithfully verify both the access control
  requirements governing specific token requests as well as the
  signatures on all incoming tokens.
\end{packed_desc}

The storage servers must uphold the following security
requirements. Failure to do so results in a failure of the security of
the storage server.

\begin{packed_desc}
\item[Token Verification:] Each storage server must securely (via
  HTTPS) obtain the public token signing key from each AC server
  delegated with providing access control for a given storage
  object. The storage server must then use these keys to faithfully
  verify the signatures on all tokens it receives. Assuming the token
  signature is valid, the storage server must faithfully enforce the
  claims asserted in a given token; e.g., by only allowing actions
  granted by the permission contained in the token on the object the
  token identifies prior to the expiration time specified by the
  token.
\item[Secure Storage:] Each storage server must take steps to store
  user-provided secrets in a secure manner, releasing them only to
  requests accompanied by the requisite number of valid tokens
  granting such release.
\end{packed_desc}

Since the tokens the storage server must verify are provided by the AC
servers, the security of the storage server with respect to a given
collection is dependent on the security of any designated AC servers
associated with said collection. If these AC servers are insecure, the
objects that delegate access to them will also be insecure.

\subsection{Security of Multiple Servers}

Unlike existing secret management systems~\cite{vault, confidant,
  keywhiz}, the Tutamen architecture is capable of remaining secure
even when individual storage or access control servers fail to meet
their security requirements. Such failures may result from physical
server compromise, software bugs, malicious intent or incompetence on
the part of the server operator, or compelled failures.\footnote{For
  example, being forced to turn over stored secrets in response to
  legal or governmental pressure.}

To work around security failures of individual server, Tutamen
applications can leverage Tutamen's distributed operation modes. In
these modes, the security of the system as a whole is diffused, no
longer relying on the security of any specific access control or
storage server in order to keep an application's secrets secure. As
described in Section~\ref{sec:tutamen:arch:distributed}, each
application can distribute both secret-storage and access control
delegations using $n$ choose $k$ schemes. In such setups, the value of
$k$ represents the degree to which a Tutamen applications can
withstand security failures of the associated AC and storage servers,
while the difference between $n$ and $k$ dictates the degree to which
an Application can withstand availability failures. For example, an
application which chooses to shard its secrets across six storage
servers where any three shards are sufficient to recreate the secret
($n=6$, $k=3$) will remain secure as long as no more than two ($k-1$)
of the storage servers fail to meet their security
obligations. Similarly, if each secret shard delegates six possible AC
servers, tokens from three of which are required to grant secret
access, the applications can withstand the failure of two AC servers
to uphold their security guarantees. In both cases, the system can
also withstand the availability failure of up to three servers ($n-k$)
while continuing to operate.

\subsection{Trust Model}

Trust in Tutamen follows from the security models of both individual
Tutamen servers and of the distributed Tutamen deployment
architectures. If a Tutamen application is leveraging only a single
storage and AC server, the application is placing a high degree of
trust in both servers (and by proxy, the operators of both
servers). This level of trust may be appropriate for some use cases
(e.g., when a user is operating their own Tutamen's servers), but is
inappropriate in many other cases (e.g., when using third party hosted
Tutamen servers). Fortunately, Tutamen allows applications to avoid
placing a high degree of trust in any single server by leveraging
multiple servers and picking $k$ and $n$ in a manner commensurate with
the degree to which each server is trusted.

Beyond minimizing the amount of trust placed in each individual
Tutamen server by leveraging multiple servers, we also envision
economic incentives helping to ensure Tutamen server
trustworthiness~\cite{anderson2001}. The Tutamen protocol is
standardized and designed to support a range of interchangeable ACS
and SS providers. Such a design allows for the development of a
Tutamen server marketplace where both ACS and SS providers can compete
against each other on the basis of trustworthiness, features (e.g.,
what types of authenticators they support), and cost. In such an
ecosystem, Tutamen service providers who fail to uphold the Tutamen
security requirements on their servers will suffer a negative economic
effect, disincentivizing such behavior. It is also likely that storage
providers who take additional steps to protect the secrets they store
(e.g., by using systems such as Trusted Platform Modules (TPMs) to
encrypt the secrets they hold and harden server security) would be
able to command a higher price in the marketplace, incentivizing such
best practices.

Thus, unlike other third-party cloud services where trustworthiness
and economic incentives are in direct competition (as is the case on
many ``free'' third part services that depend on selling user data in
order to generate revenue)~\cite{flowerday2006}, Tutamen encourages a
system where economic incentives are well aligned with user
trust. That fact, coupled with the high degree of control over
third-party trust Tutamen grants to each application by allowing each
application to select how many servers to diffuse trust across, make
Tutamen a robust system in the face of both security and
trustworthiness failures. Such robustness is a critical component of
any successful secret storage system.

%%  LocalWords:  Tutamen's Tutamen CAs DNS TPMs authenticators HTTPS
%%  LocalWords:  ACS
