\section{Security and Trust}
\label{sec:trust}

One of Tutamen's primary design goals is its ability to support a wide
range of security and trust requirements. It achieves this goal
through its support for both centralized and distributed operation as
well as though its support for a range of authentication mechanisms.

\subsection{Security of Individual Servers}

The Tutamen security model is rooted in the security of the access
control servers. If the operation of these servers is secure, the
operation of the rest of the system may also be secure. Any security
guarantees provided via the access control server are extended to the
storage servers via the signed tokens the AC servers provide.

The security of each individual access control server rests on several
requirements. Failure to uphold these requirements will result in the
failure of any security guarantees provided by the AC server.

\begin{packed_desc}
\item[Certificate Authority Role:] Each access control server acts as
  a CA delegated with issuing and verifying client certificates. Thus,
  it's important that the CA keys stored by each AC server remain
  secure and that each CA server faithfully verifies the certificate
  presented by each client connection. Failure in either of these
  roles will result in a failure of any security guarantees provided
  by the AC server.
\item[Token Issuance and Verification:] Each access control server is
  responsible for verifying the access control requirements bound to
  specific object/permission combinations, issuing signed tokens
  attesting to such verification, and verifying the signatures of the
  tokens it receives from clients wishing to operate on access control
  objects. As such, it's important that each AC server keep its
  private token signing key secure and that if faithfully verifies
  both the access control requirements governing specific token
  requests as well as the signatures on all incoming tokens.
\end{packed_desc}

The storage servers must uphold the following security
requirements. Failure to do so results in a failure of the security of
the storage server.

\begin{packed_desc}
\item[Token Verification:] Each storage server must obtain the public
  token signing key from each AC server delegated with providing
  access control for a given storage object. The storage server must
  then use these keys to faithfully verify the signatures on all
  tokens it receives. Assuming the token signature is valid, the
  storage server must faithfully enforce the claims asserted in a
  given token: e.g. by only allowing actions granted by the permission
  contained in the token on the object the token identifies prior to
  the expiration time specified by the token.
\item[Secure Storage:] Each storage server must take steps to store
  user-provided secrets in a secure manner, releasing them only to
  requests accompanied by the requisite number of valid tokens
  granting such release.
\end{packed_desc}

Since the tokens the storage server must verify are provided by the AC
servers, the security of the storage server with respect to a given
collection is dependent on the security of any designated AC servers
associated with said collection. If these AC servers are insecure, the
storage server will also be insecure.

%% The security of Tutamen, in most cases, depends on the security
%% of the public CA/PKI system. While AC servers act as local CAs for the
%% purpose of issuing and verify client certificates, we believe that in
%% most cases both Tutamen AC and storage servers will rely on public CA
%% system for obtain server certificates verifying the mapping between
%% DNS names and the server itself -- as do most public websites. Since
%% Tutamen storage servers retrieve the correct signing keys form each AC
%% server on the basis of the AC servers domain name, properly verifying
%% this name is critical. Similarly, its important for clients to verify
%% they are talking to the storage and AC servers they intend to, and
%% verifying the server certificates of each of these servers via the
%% standard TLS handshake process if thus also critical.

%Todo: cite other TLS security efforts such as CT, etc?

\subsection{Security of Multiple Servers}

Unlike existing secret management systems~\cite{vault, confidant,
  keywhiz}, the Tutamen architecture is capable of remaining secure
even when individual storage or access control servers fail to meet
their security requirements. Such failures may result from physical
server compromise, software bugs, malicious intent or incompetence on
the part of the server operator, or compelled failures.\footnote{For
  example, being forced to turn over stored secrets in response to
  legal or governmental pressure.}

To work around security failures of individual server, Tutamen
applications can leverage Tutamen's distributed operation modes. In
these modes, the security of the system as a whole is diffused -- no
longer relying on the security of any specific access control or
storage server in order to keep an applications' secrets secure. As
described in Section~\ref{sec:tutamen:arch:distributed} each
application can distribute both secret storage and access control
delegations using $n$ choose $k$ schemes. In such setups, the
difference between $n$ and $k$ represents the degree to which a
Tutamen applications can withstand security failures of the associated
AC and storage servers. For example, an applications which chooses to
shard its secrets across five storage servers where any three shards
are sufficient to recreate the secret ($n=5$, $k=3$) will continue to
remain secure even if two of the storage servers fail to meet their
security obligations. Similarly, if each secret shard delegates five
possible AC servers, tokens from three of which are required to grant
secret access, the applications can withstand the failure of two AC
server to uphold their security guarantees.

%% Since the security of each storage server is dependent on the security
%% of the associated AC servers, in most cases\footnote{There may be
%%   cases where a user happens to trust the AC servers more than the
%%   storage servers that would justify having a larger $k-n$ storage
%%   delta than the $k-n$ access control delta.} there is no security
%% benefit from having a greater $n-k$ delta for storage servers than for
%% AC servers (although such a delta may still yield reliability
%% benefits).

\subsection{Trust Model}

Trust in Tutamen follows from the security models of both individual
Tutamen servers and of the distributed Tutamen deployment
architectures. If a Tutamen application is leveraging only a single
storage and AC server, the application is placing a high degree of
trust in both servers (and by proxy, the operators of both
servers). This level of trust may be appropriate for some use cases --
e.g. when a user is operating their own Tutamen's servers -- but is
inappropriate in many other cases (e.g. when using third party hosted
Tutamen servers). Fortunately, Tutamen also allows applications to
avoid placing a high degree of trust in any single server by
leveraging multiple servers and picking $k$ and $n$ in a manner that
corresponding to the degree by which server is trusted.

Beyond minimizing the amount of trust placed in each individual
Tutamen server by leveraging multiple servers, we also envision
economic incentives helping to ensure Tutamen
server trustworthiness. The Tutamen protocol is standardized and
designed to support a range of interchangeable AC and storage server
providers. Such a design allows for the development of a Tutamen
server marketplace where both AC and storage server providers can
compete against each other on the basis of trustworthiness, features
(e.g. what types of authenticates they support), and cost. In such an
ecosystem, Tutamen service providers who fail to uphold the Tutamen
security requirements on their servers will suffer a negative economic
effect, disincentivizing such behavior. Its also likely the storage
providers who take additional steps to protect the secrets they store
(e.g. by using system such as Trusted Platform Modules (TPMs) to
encrypt the secrets they hold and harden server security) would be
able to command a higher price in the marketplace, incentivizing such
best practices.

Thus, unlike other third-party cloud services where faithfulness to
customer's desires and economic incentives are in direct competition
(e.g. as is the case on many ``free'' third part services that depend
on selling user data to advertisers in order to generate revenue),
Tutamen encourages a system where economic incentives are well aligned
with applications' security. That fact, coupled with the high degree
of control over third-party trust Tutamen grants to each application
by allowing each application to select how many servers to diffuse
trust across, make Tutamen a robust system in the face to both
security and trustworthiness failures. Such robustness is a critical
component of any successful secrete storage system~\cite{anderson2001,
  flowerday2006}.


%%  LocalWords:  Tutamen's Tutamen CAs DNS TPMs
