\section{Conclusion}
\label{sec:conclusion}

How best to securely store secrets is a pressing issue in today's
cloud-oriented, third-party hosted, ephemeral infrastructure adopting
environment. This need has triggered the creation of several
secret-storage frameworks and systems. Unfortunately, these existing
systems prove deficient in at least two key secret storage
capabilities: the ability to be operate atop untrusted infrastructure
and the ability to support a wide range of use cases -- including
autonomous access use cases or out-of-band human-in-the loop
verification use cases.

We created Tutamen to demonstrate our concept for a next generation
secret storage system. Tutamen supports client-controlled secret
sharding to allow applications to leverage minimally-trusted server
infrastructure. Tutamen also supports a flexible and modular
authentication mechanism that allows end-users to specify complex
access control requirements. We've successfully coupled Tutamen with a
number of applications, including full disk encryption on headless
servers and client-side encrypted file sharing between multiple
parties. These use cases would be difficult (or at least burdensome on
the end-user) to realize without a system such as Tutamen. We feel
Tutamen represents the way forward in addressing the secret storage
problem.

Our Tutamen prototype has proven quite useful in each use case where
we have leveraged it, and we plan to continue developing the Tutamen
ecosystem. On the server side, we have plans to work toward increased
performance and to add support for additional authenticator
modules. While the Tutamen servers currently have basic logging, we
also plan to expand this support and explore interfacing Tutamen audit
logs with intrusion detection systems with an aim toward exposing more
actionable intelligence to Tutamen authenticator modules. We are also
considering tying Tutamen's logging infrastructure to a publicly
audit-able system similar to the Certificate
Transparency~\cite{laurie2013} project. Such a system would help to
further reduce the trust Tutamen users must place in individual
Tutamen servers by exposing a mechanisms by which such user could
reliably audit the behavior of such providers.

We have made all of the Tutamen source code available via the
previously referenced repositories and Free Software licenses. We
encourage others to experiment with our Tutamen prototype and
reference implementation, or to integrate Tutamen with their projects
or applications. We hope that Tutamen (or similar secret storage
systems) can help to ease the burden currently imposed on
administrators, developers, and end users who must otherwise undertake
the significant burden of manually managing sensitize material such as
file encryption keys, passwords, and other secrets.

%%  LocalWords:  Tutamen authenticator Tutamen's
