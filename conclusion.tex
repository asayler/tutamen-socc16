\section{Conclusion}
\label{sec:conclusion}

How best to securely store secrets is a pressing issue in today's
cloud-oriented, third-party hosted, ephemeral-infrastructure adopting
environment. This need has triggered the creation of several
secret-storage frameworks and systems. Unfortunately, these existing
systems prove deficient in at least three key secret-storage
capabilities: the ability to support operation outside of a single
administrative domain, the ability to operate atop untrusted
infrastructure, and the ability to support a wide range of use cases.

We created Tutamen to demonstrate our concept for a next-generation
secret-storage system. Tutamen supports client-controlled secret
sharding to allow applications to leverage minimally-trusted server
infrastructure. Tutamen also supports a flexible and modular
authentication mechanism that allows end users to specify complex
access control requirements. We've successfully coupled Tutamen with a
number of applications, including full disk encryption on headless
servers and client-side encrypted file sharing between multiple
parties. These use cases would be difficult (or at least burdensome)
to realize without a system such as Tutamen.

We plan to continue developing the Tutamen ecosystem. On the
server-side, we have plans to work toward increased performance and to
add support for additional authenticator modules. While the Tutamen
servers currently have basic logging support, we also plan to expand
this support, and to explore interfacing Tutamen audit logs with
intrusion detection systems in order to expose more actionable
intelligence to Tutamen authenticator modules. We are also considering
tying Tutamen's logging infrastructure to a public audit system
similar to~\cite{laurie2013}. Such a system would help to further
reduce the trust Tutamen users must place in individual Tutamen
servers by exposing mechanisms by which a user could reliably audit
the behavior of such providers.

We have made all of the Tutamen source code available via the
previously referenced repositories. We encourage others to experiment
with our Tutamen prototype and reference implementation, or to
integrate Tutamen with their projects or applications. We hope that
Tutamen (or similar secret-storage systems) can help to ease the
secret-storage burden currently imposed on administrators, developers,
and end users by providing an alternative to manually managing
sensitive secrets in a manner that also minimizes third party trust.

%%  LocalWords:  Tutamen authenticator Tutamen's
