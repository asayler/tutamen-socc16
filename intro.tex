\section{Introduction}
\label{sec:intro}

How best to store and manage secrets -- the bits of tightly controlled
data necessary to ensure or bootstrap and security of larger data
stores or computing systems -- has always been a non-trivial
problem. As we continue to move toward computing and storage platforms
controlled by third parties and embrace modern desires for ephemeral
infrastructure, the secret storage problems only becomes more
prevalent and important to answer. Tutamen\footnote{Latin -- A means
  of protection or defense.} is our attempt to solve the secret
storage problem in a manner that allows the user to adhere to a range
of security and privacy requirements.

\subsection{The Need for Secret Storage}

The pervasiveness of computing systems today invades every contour of
our lives -- from the fitness trackers on our wrists to our ``smart''
home appliances to the server infrastructure required to support the
range of web sites and services we interact with every day. With this
explosion of in computing systems has come an equally large explosion
in the amount of data stored by and about us. While some of this data
is designed to be public (i.e. the entries on Wikipedia), much of it
is not - requiring the enforcement of various privacy and security
guarantees with respect to its handling and storage. The basis of
proving such guarantees relies on our ability to keep and selectively
share secrets - from encryption keys used to encrypt such data or
serve it over a secure connection, to the passwords used to protect
our online accounts, to personally sensitive data such as social
security numbers or other commonly used human authentication
information.

How best to store and manage these secrets is thus a significant
question that provides the foundation to all of computing's higher
level security and privacy guarantees. Beyond the repaid expansion of
sensitive digital data in existence there are several other factors
driving the need to good solution to the secret storage problem.

On the developer front, the trend toward ephemeral infrastructure
capable of being rapidly scaled up or down as demand requires is
driving the need for suitable configuration management systems capable
of converting configuration metadata into fully configured
systems. Systems such as puppet~\cite{puppet} or chef~\cite{chef}
allow administrators to manage and configure large fleets of ephemeral
systems with a high degree of efficiency. Such systems, however, do
not tend to have suitable mechanisms for providing the security and
privacy requirements inherent to storing and controlling access to
configuration data that must remain secret. And yet configuration data
often contains a variety of secrets including:

\begin{packed_item}
\item Cryptographic secrets such a SSH keys, TLS/SSL keys, and file
  encryption keys
\item API Tokens or credentials necessary to authenticate to a variety of
  API endpoints
\item Others?
\end{packed_item}

On the end-user front, the need for suitable secret storage systems is
being driven by the rapid expansion in the number of sites and
services to which user must authenticate themselves as well as by the
growing expanse of private digital data users wish to protect. Indeed,
the popularity of password management systems such as
LastPass~\cite{lastpass} or 1Password~\cite{onepassword} as well as
users desire for encryption support on their
devices~\cite{intercept-cookencryption} demonstrate the importance of
secret storage to end users.

Unlike standard configuration management systems, or even specific
secret storage systems such as password managers, a general purpose
secret storage presents a number of unique requirements including:

\begin{packed_item}
\item The ability to store a wide-range of secret data
\item The ability to store data in a secure manner
\item The ability to enforce fine-grained access control requirements
\item The ability to support a range of authentication sources and methods
\item The ability to provide audit logs tracking data access history
\end{packed_item}

In response to these needs, a number of general purpose secret-storage
systems have been proposed or developed including HashiCorp's
Vault~\cite{vault}, Lyft's Confidant~\cite{confidant}, and Square's
Keywhiz~\cite{keywhiz}.

\subsection{The Ideal Secret Storage System}



%%  LocalWords:  Tutamen LastPass HashiCorp's Lyft's Keywhiz
