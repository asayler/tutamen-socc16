\section{Introduction}
\label{sec:intro}

How best to store and manage secrets -- the bits of tightly controlled
data necessary to ensure or bootstrap the security of computing
systems and services -- has always been a non-trivial problem. As we
continue to move toward computing and storage platforms controlled by
third parties, and embrace modern trends toward ephemeral
infrastructure, the secret storage problem only becomes more prevalent
and critical to solve.

Tutamen\footnote{Latin -- A means of protection or defense.} is our
attempt to solve the secret storage problem in a manner that allows
the user to adhere to a range of security and privacy requirements
without sacrificing functionality in the process. Tutamen is a next
generation secret storage platform. It builds on our previous secret
storage efforts~\cite{custos-trios} and strives to offer features not
provided by the various secret storage systems currently in
use. In this paper, we use Tutamen, we offer several contributions:

\begin{packed_item}
\item Built-in support for modular authentication modules designed to
  enable the use of contextual and alternate-band (e.g. SMS text
  messages) authentication mechanism
\item Ability to operate atop minimally trusted infrastructure
  (i.e. Tutamen requires no single ``point-of-trust'' outside of the
  applications that leverage it). This includes the ability to
  leverage multiple storage and access control providers to achieve
  both redundancy and to mitigate trust.
\item Several practical demonstrations of how a secret storage system
  such as Tutamen can be integrated with real-world applications to
  offer secure desirable features in a secure and easy-to-use manner.
\end{packed_item}

\subsection{The Need for Secret Storage}

Computing systems today invade every contour of our lives -- from the 
fitness trackers on our wrists, to our ``smart'' home appliances, to 
the server infrastructure required to support the range of web sites 
and services we interact with every day. With this explosion of 
computing systems has come an equally large explosion in the amount of 
data stored by and about us. While some of this data is designed to be 
public (i.e. the entries on Wikipedia), much of it is not -- requiring 
the enforcement of various privacy and security guarantees with 
respect to its handling and storage. The basis of proving such 
guarantees relies on our ability to store and selectively share 
secrets ranging from the keys used to encrypt our data, or serve it 
over a secure connection, to the passwords used to protect our online 
accounts. How best to store and manage these secrets is thus a
significant question -- the answer to which forms the foundation to
all of computing's higher level security and privacy guarantees.

Beyond the need to bootstrap a variety of security guarantees, there
are several other factors driving the need for secrete storage
solution. On the system administration front, the trend toward
ephemeral infrastructure cable of rapidly scaling up or down is
driving the adoption of configuration management systems such as
puppet~\cite{puppet} or chef~\cite{chef}. Such systems, however, do
not tend to have suitable mechanisms for enforcing the security and
privacy requirements inherent to storing secrets. Despite this,
configuration data often contains a variety of secrets including
cryptographic secrets such as SSH keys, TLS/SSL keys, and disk or file
encryption keys, as well as tokens or credentials necessary to
authenticate to a variety of external APIs and services.

Similarly, on the end-user front, the need for suitable secret storage
systems is being driven by rapid expansion in the number of sites and
services to which users must authenticate themselves, as well as by
the growing expanse of digital data users wish to protect. Indeed, the
popularity of password management systems such as
LastPass~\cite{lastpass} or 1Password~\cite{onepassword} as well as
users desire for encryption support on their
devices~\cite{intercept-cookencryption} demonstrate the importance of
secret storage, and the applications it enables, to end users.

\subsection{The Ideal Secret Storage System}

Unlike standard configuration management systems, or even specific
secret storage systems such as password managers, a general purpose
secret storage presents a number of unique requirements including:

\begin{packed_item}
\item The ability to store a wide-range of arbitrary secret data
\item The ability to store data in a secure manner
\item The ability to enforce fine-grained access control requirements
\item The ability to support a range of authentication sources and methods
\item The ability to provide audit logs tracking data access history
\end{packed_item}

In response to these needs, a number of general purpose secret-storage
systems have been recently developed, including HashiCorp's
Vault~\cite{vault}, Lyft's Confidant~\cite{confidant}, and Square's
Keywhiz~\cite{keywhiz}. These systems exist to fulfill some or all of
the requirements listed above. We believe, however, that such systems
are hindered by several key limitations. First, they generally require
at least one trusted system as the basis of their security model,
making them unsuitable for operation atop untrusted
infrastructure. Second, they tend to lack support for use cases
requiring autonomous or remote access to secret material in a secure
manner. These deficiencies give rise to a few more secret
storage requirements:

\begin{packed_item}
\item Avoidance of needing to place a high degree of trust in any
  single system outside the client application that wishes to store a
  secret.
\item Ability to support a range of secret access use-cases -
  including use cases where automatic or remote access may be required.
\end{packed_item}

It is toward these final two requirements that Tutamen attempts to
provide advancements over existing secret storage systems. In
particular, Tutamen supports operational modes where no single entity
other than the client must be trusted. This allows users to leverage
third party secret storage providers running Tutamen servers without
having to place high degrees of trust in any single provider. Tutamen
has also been developed with support for a modular authentication
interface. This interface makes Tutamen suitable for use in situations
where it is desirable to leverage external environmental information
to to keep a human in the authentication loop without actually
requiring that the human be physically present or connected to the
application requesting a secret. For example, Tutamen can be used to
store the disk encryption keys required to boot a physical server of
virtual machines, and only release these keys to the server requesting
them when a human responds to a text message confirming the boot
request.

% Todo: comparison table?

%%  LocalWords:  Tutamen LastPass HashiCorp's Lyft's Keywhiz OOB SMS
