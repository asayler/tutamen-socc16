\section{Introduction}
\label{sec:intro}

How best to store and manage secrets -- the bits of tightly controlled
data necessary to ensure or bootstrap the security of larger data
stores or computing systems -- has always been a non-trivial
problem. As we continue to move toward computing and storage platforms
controlled by third parties and embrace modern trends toward ephemeral
infrastructure, the secret storage problems only becomes more
prevalent and critical to solve.

Tutamen\footnote{Latin -- A means of protection or defense.} is our
attempt to solve the secret storage problem in a manner that allows
the user to adhere to a range of security and privacy requirements.
Tutamen is a next generation secret storage platform. It builds on our
previous secret storage efforts~\cite{custos-trios} and strives to
offer features not provided by the various secret storage systems
previously proposed. In particular, Tutamen focuses on two key secret
storage features absent form other systems -- the ability to support
contextual and out-of-band (OOB) authentication mechanisms and the
avoidance of any single-point-of-trust requirements.

\subsection{The Need for Secret Storage}

The pervasiveness of computing systems today invades every contour of
our lives -- from the fitness trackers on our wrists, to our ``smart''
home appliances, to the server infrastructure required to support the
range of web sites and services we interact with every day. With this
explosion of computing systems has come an equally large explosion in
the amount of data stored by and about us. While some of this data is
designed to be public (i.e. the entries on Wikipedia), much of it is
not - requiring the enforcement of various privacy and security
guarantees with respect to its handling and storage. The basis of
proving such guarantees relies on our ability to store and selectively
share secrets - from encryption keys used to encrypt our data or
serve it over a secure connection, to the passwords used to protect
our online accounts, to personally sensitive data such as social
security numbers or other commonly used human authentication
information.

How best to store and manage these secrets is thus a significant
question -- the answer to which forms the foundation to all of
computing's higher level security and privacy guarantees. Beyond those
mentioned above, there are several other factors driving the need for
good solution to the secret storage problem.

On the developer front, the trend toward ephemeral infrastructure
capable of being rapidly scaled up or down is driving the need for
suitable configuration management systems capable of converting
configuration metadata into fully configured systems. Systems such as
puppet~\cite{puppet} or chef~\cite{chef} allow administrators to
manage and configure large fleets of ephemeral systems with a high
degree of efficiency. Such systems, however, do not tend to have
suitable mechanisms for providing the security and privacy
requirements inherent to storing and controlling access to
configuration data that must remain secret. And yet configuration data
often contains a variety of secrets including:

\begin{packed_item}
\item Cryptographic secrets such a SSH keys, TLS/SSL keys, and file
  encryption keys
\item API Tokens or credentials necessary to authenticate to a variety of
  API endpoints
% Todo: others>?
\end{packed_item}

On the end-user front, the need for suitable secret storage systems is
being driven by the rapid expansion in the number of sites and
services to which user must authenticate themselves as well as by the
growing expanse of private digital data users wish to protect. Indeed,
the popularity of password management systems such as
LastPass~\cite{lastpass} or 1Password~\cite{onepassword} as well as
users desire for encryption support on their
devices~\cite{intercept-cookencryption} demonstrate the importance of
secret storage to end users.

\subsection{The Ideal Secret Storage System}


Unlike standard configuration management systems, or even specific
secret storage systems such as password managers, a general purpose
secret storage presents a number of unique requirements including:

\begin{packed_item}
\item The ability to store a wide-range of secret data
\item The ability to store data in a secure manner
\item The ability to enforce fine-grained access control requirements
\item The ability to support a range of authentication sources and methods
\item The ability to provide audit logs tracking data access history
\end{packed_item}

In response to these needs, a number of general purpose secret-storage
systems have been proposed or developed including HashiCorp's
Vault~\cite{vault}, Lyft's Confidant~\cite{confidant}, and Square's
Keywhiz~\cite{keywhiz}. These systems exist to fulfill some or all of
the requirements listed above. We believe, however, that such existing
systems are hindered by several key limitations: they generally
require at least one trusted system as the basis of their security
model and they tend to lack support for use cases requiring autonomous
access to secret material in a secure manner. These give rise to a few
more ideal secret storage system requirements:

\begin{packed_item}
\item Avoidance of needing to place a high degree of trust in any
  system outside the client itself
\item Ability to securely support a range of secret access use-cases -
  including use cases where automatic access may be required
\end{packed_item}

It is toward these final two requirements that Tutamen attempts to
provide advancements over existing secret storage systems. In
particular, Tutamen support operational modes where no single entity
other than the client must be trusted. This allows users to leverage
third party secret storage providers running Tutamen servers with
having to place high degrees of trust in any single provider. Tutamen
has also been developed with support for a module out-of-band
authentication interface. This interface makes Tutamen suitable for
use in situations where it is desirable to keep a human in the
authentication loop without actually requiring that the human be
physically present or connected to the client requesting a secret. For
example, Tutamen can be used to store the disk encryption keys
requires to boot s physical server of virtual machine and to only
release these keys to the servers requesting them via the boot process
when a separate human responds to a text message confirming the boot
request.

% Todo: comparison table?

%%  LocalWords:  Tutamen LastPass HashiCorp's Lyft's Keywhiz OOB
